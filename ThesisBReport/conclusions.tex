\section{Conclusions}
The suite of simulations showed that constellations with higher altitude or a higher number of satellites provided more raw coverage time. A higher altitude allowed for a greater sensor footprint, meaning that a trans-oceanic flight would stay in-view for longer. Having more satellites increased the probability of a satellite being above a trans-oceanic flight at any one point in time. The effect of this was significant enough such that the penalty of the increased constellation cost was offset by the net benefit to the coverage opportunities.

Despite the improved geographic coverage performance of high altitude constellations, low altitude constellations showed a marked improvement in signal performance. Less distance was required to travel by any one ADS-B transmission at lower altitudes, resulting in lower signal loss. This effect was dramatic enough such that the lowest altitude constellation performed better than most other constellations as seen in Table \ref{tab:decMatRes}.

The inclination trend analysis suggested that a constellation with all satellites inclined at 90 degrees may represent the best coverage trade-off between the three flights. This inclination represented the best possible geometric configuration for full global coverage. However, the nature of one-dimensional parameter variance in the tests conducted and the weighted decision matrix meant that the 90 degree inclination configuration did not feature in the `best choices', with system coverage overriding the potential cost of the system as a performance metric. 
\subsection{Chosen Constellation}
The results from the weighted decision matrix show that a satellite constellation with parameters as defined in Table \ref{tab:18sat_winner} is the best performing space based ADS-B system of those tested. This particular constellation performed strongly in being able to provide a high level of coverage for the three trans-oceanic flights studied in this thesis. A 3D model of this constellation is shown in Figure \ref{fig:18sat_3D} and the ground tracks in Figure \ref{fig:18sat_groundtrack}

% Table generated by Excel2LaTeX from sheet 'Sheet2'
\begin{table}[htbp]
  \centering
  \caption{18 satellite configuration with the best score}
    \begin{tabular}{lr}
    \toprule
    Parameter & Value \\
    \midrule
    Altitude (km above mean radius of Earth) & 700 \\
    Inclination (deg) & 60 \\
    Number of Planes & 3 \\
    Plane Separation (deg RAAN) & 120 \\
    Number of Satellites per plane & 6 \\
    True Anomaly Separation (deg) & 60 \\
    Number of Satellites (Total) & 18 \\
    \bottomrule
    \end{tabular}%
  \label{tab:18sat_winner}%
\end{table}%

\begin{figure}[htbp]
	\centering
	\includegraphics[scale = 0.4]{Pictures/18sat_3D.png}
	
	\caption{18 satellite configuration rendered in STK showing a) the view above North America and b) the view above the Arctic}
	\label{fig:18sat_3D}
\end{figure}

\begin{figure}[htbp]
	\centering
	\includegraphics[scale = 0.6]{Pictures/18sat_groundtrack.png}
	
	\caption{Ground track of the 18 satellite configuration, rendered in STK}
	\label{fig:18sat_groundtrack}
\end{figure}

This particular result shows that a higher-number of satellites performs more favourably when considering ADS-B coverage requirements. Interestingly, a fewer satellites in a lower orbit performed almost as well due to dramatically increased communication link quality. In this case, the benefit of having more effective coverage outweighed the benefit of having less satellites. This may not be the case if the decision matrix was weighted more heavily to budgetary considerations.  The tests conducted only examined variation of one parameter at a time and can not provide a conclusion on whether a combination of two or more parameters would result in better performance.
