\section{Model Input Data and Assumptions}
Data used in the simulations was researched from official sources was used in order to generate simulations that would closely represent the expected true-to-life situations. 
\subsection{Link Type}
A series of simplifying assumptions were made about the characteristics of the ADS-B links being analysed. The parameters used in the STK link model are summarised in Table \ref{tab:linkParams}. 
% Table generated by Excel2LaTeX from sheet 'Sheet1'
\begin{table}[H]
  \centering
  \caption{Link Parameters used in STK}
    \begin{tabular}{p{3cm}rp{5cm}r}
    \toprule
    \textbf{Parameter} & \textbf{Value} & \textbf{Description} & \textbf{Source} \\
    \midrule
    Transmitter Model & `Simple Transmitter Model' & The transmitter model type used by STK to propagate and analyse link budgets & \cite{STKOnline} \\ \hline
    Equivalent Isotropically Radiated Power (EIRP) & 240W & Power transmitted through an ADS-B antenna assuming zero losses between transponder and antenna & \cite{Garmin2007,Corporation2011,TrigAvionics,BendixKing2013}  \\ \hline
    Bit Error Rate (BER) & $10^{-6}$ & Minimum bit error rate of a link &  \cite{RTCA2013}  \\ \hline
    Modulation Type & `Pulsed Signal' & Type of modulation used in the link &  \\ \hline
    Bandwidth & 8MHz & The bandwidth of an ADS-B Signal &  \cite{RTCA2013}  \\ \hline
    Frequency & 1090MHz & The carrier frequency of ADS-B and Mode S signals &  \cite{STKOnline} \\ \hline
    Receiver Model & `Simple Receiver Model' & The receiver model type used by STK to analyse links from transmitters &  \cite{STKOnline} \\ 
    \bottomrule
    \end{tabular}%
  \label{tab:linkParams}%
\end{table}%

Transmission power was generalised based upon a survey of commercially available transponders \cite{Garmin2007,Corporation2011,TrigAvionics,BendixKing2013} to 240W, transmitted with the assumption that there were insignificant signal losses between the transmitter and antenna. The minimum bit-error rate (BER) was set to $10^6$ bits per second based on the ADS-B  transimission specification released by the RTCA \cite{RTCA2013}. A `pulsed signal' was used in order to model the radar-like behaviour of ADS-B, as suggested by \cite{STKOnline} and no filter was applied to the output. The signal bandwidth was set to 8MHz (+/- 4MHz) as specified by \cite{RTCA2013}. 

These parameters were input to a `Simple Transmitter Model' was used in STK, as the suggested model to use during the system engineering process \cite{STKOnline}. Modelling a more complex transmitter model and defining antenna and propagation type were considered outside the scope of this thesis.

Each of the satellites in the space segment was given a `Simple Receiver Model'. The receiver gain divided by system noise temperature $G/T$ was left unchanged from the default $20K$. Other link parameters would be automatically completed upon analysis of a link with a specified transmitter \cite{STKOnline}. 

\subsection{Flight Characteristics}
Analysis focussed primarily on the performance of ADS-B coverage for a plane during the transoceanic portion of the flight path of interest. Planes within range of an airport would already be serviced by the airport's ground ADS-B receivers. Flights were therefore modelled as single aircraft travelling at a constant cruise speed and altitude between the source and destination airports, ignoring landing and take-off procedures.  

Typical cruise altitudes for commercial airliners range between 30 000 and 40 000 feet. The lower bound of 30 000 feet was chosen in order to maximise the distance travelled by an ADS-B transmission, producing a worst case scenario for signal loss.

The Boeing 747 was chosen as a `typical' commercial transoceanic aircraft, with a cruising speed of 913 km/h. The 747 was chosen after examining a cross section of the typical cruising speeds of popular `long haul' commercial aircraft,  given in Table \ref{tab:flightSpecs}.

% Table generated by Excel2LaTeX from sheet 'Sheet2'
\begin{table}[H]
  \centering
  \caption{Cruising speeds of typical commercial aircraft}
    \begin{tabular}{lrrrr}
    \toprule
    Manufacturer & Model & Speed (km/h) & Speed(km/s) & Source \\
    \midrule
    Boeing & 747-400 & 913   & 0.25361 & \cite{Boeing747}  \\
    Boeing & 777-200 & 1029 & 0.28584 &  \cite{Boeing777} \\
    Boeing & 757-200 & 980 & 0.27222 & \cite{Boeing757}  \\
    Airbus & A380  & 945   & 0.2625 &  \cite{Frawley2014} \\
    \bottomrule
    \end{tabular}%
  \label{tab:flightSpecs}%
\end{table}%

A summary of the aircraft parameters used in STK is given in Table \ref{tab:flightSTKParams}

\begin{table}[H]
  \centering
  \caption{STK Parameters used for each flight model}
    \begin{tabular}{lr}
    \toprule
    Parameter & Value\\
    \midrule
    Altitude & 9.144 km  \\
    	& (30,000 feet)\\
    Speed & 0.2536 km/sec \\
    	& (913 km/h) \\	
    \bottomrule
    \end{tabular}%
  \label{tab:flightSTKParams}%
\end{table}% 

The global co-ordinates for the 4 airports of interest were taken from \cite{Open}, presented in Table \ref{tab:airportCoords}.

\begin{table}[H]
  \centering
  \caption[Airport global co-ordinates]{Airport global co-ordinates, from \cite{Open}.}
    \begin{tabular}{p{4cm}lrrr}
    \toprule
    Airport Name & City & Code & Longitude & Latitude \\
    \midrule
    Los  Angeles International Airport & Los Angeles & LAX   & 33.9425$^\circ$N & 118.4081 $^\circ$W  \\
    London Heathrow Airport & London & LHR   & 51.4775$^\circ$N & 0.4614 $^\circ$W  \\
   	Narita International Airport & Tokyo & NRT   & 35.7653$^\circ$N & 149,3856 $^\circ$E   \\
    Sydney Airport & Sydney & SYD   & 33.9461$^\circ$S & 151,1772 $^\circ$E  \\
    \bottomrule
    \end{tabular}%
  \label{tab:airportCoords}%
\end{table}%

\subsection{Satellite Characteristics}
The standard STK satellite model was used to model each satellite in the test constellations. It was assumed that there would be attitude control aligning the antenna element with the nadir and no active station keeping systems. The STK \Verb|J4Pertubation| orbit propagator was used in order to account for the oblateness of the Earth.