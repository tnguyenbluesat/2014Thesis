\chapter{Introduction}
Automatic Dependant Surveillance Broadcast (ADS-B) is quickly being adopted as the standard method for Air Traffic Control (ATC) Management in Europe, Australia and the United States. The currently available methods that provide ADS-B coverage to Air Navigation Service Providers (ANSPs) and ATC towers rely on terrestrial antennae which operate on line-of-sight. This presents a problem as aircraft cannot be tracked via ADS-B over oceanic and polar regions, where installing ADS-B compatible ground stations is not practical. The recent development of the lost Malaysian Airlines flight MH370 on March 8th, 2014 highlights the detriment of not having constant, global, real-time coverage of commercial aircraft. These coverage gaps can be closed with the implementation of a space-based ADS-B receiver system.

This thesis explores the possibility of implementing a space-based ADS-B system on a constellation of Low-Earth-Orbit (LEO) CubeSats. There is ongoing research in the implementation of ADS-B systems from LEO, focussing on developing and evaluating the performance of single receiver units. This thesis presents a parametric study into the performance of different CubeSat constellation configurations when used to track trans-oceanic flights. The evaluated trans-oceanic ADS-B coverage will be compared against changing orbital elements of different satellites in order to determine the `best possible' ADS-B satellite constellation. 

\section{Problem Statement}
A satellite-based ADS-B system will need to provide coverage to flights that are out of range of terrestrial ADS-B receivers. Geographically, this means that the proposed constellation needs to provide line-of-sight access to planes flying transoceanic or polar flight paths. The constellation also needs to provide enough coverage time to properly detect all flights in the area of interest and generate regular `updates' for ANSPs to properly track any flight at any given time. The constellation will need to have a communications architecture that performs reliably enough receive and decode ADS-B signals from LEO. In addition, the subsequent satellite technology requirements produced from any acceptable constellation should take into account the technological limitations of the CubeSat design form.

\section{Thesis Structure}
This work presented in this thesis is separated into four chapters. Background theory and a review of ongoing research into space based ADS-B systems is presented in Chapter \ref{part:lit_review}. The design of the experiments used to simulate the performance of potential ADS-B satellite constellations is presented in Chapter \ref{part:exp_method}. A analysis of these results and the performance of each constellation is presented and compared in Chapter \ref{part:results}, with a special case study for the MH370 flight given in Section \ref{sec:mh370}. Finally a set of conclusions and recommendations for further analysis and study is given in Chapter \ref{part:concl}.