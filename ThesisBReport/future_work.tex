\section{Future Work}
The following work should be conducted in order to fully develop a set of mission and system specifications for a CubeSat based ADS-B system.
\subsection{Full Parametric Study}
The findings of this thesis are as a result of one-dimensional parameter studies. As such they are indicative of general trends observed when varying only one constellation parameter at a time. This only gives an indication of what the best possible satellite configuration could be. In order to determine the global best performer, a full n-dimensional parametric study should be carried out, simulating all possible variations of all the orbital parameters. The simulations created for this thesis form a good basis from which a full parametric set of simulations can be generated.

\subsection{Transmitter Receiver Model Improvements}
As noted in Section \ref{sec:perfMetrics}, the simple transmitter and receiver models used produced an unrealistic link-budget characterisation for ADS-B communication links. Although received isotropic power is a good metric to compare constellation performance, the link budget results in this thesis do not represent realistic system behaviour. Incorporating a more sophisticated antenna design and better radio-frequency front-end into the communication models would more accurately characterise ADS-B signal performance. This would be necessary in further systems analysis and design.

\subsection{Ground Link}
Further study needs to be performed into how to establish a regular downlink between the satellites in the constellation and a ground station. The regularity of this downlink would determine the `total system update rate' or the speed with which data on all detected flights can be disseminated to ANSPs and therefore general use. Potential methods could be to
\begin{itemize}
	\item Use network of community-run amateur radio ground stations
	\item Transmit downlink packets to currently in-orbit communication constellations, such as Iridium or Globalstar
	\item Relay raw ADS-B data to airports and terrestrial ADS-B receivers.
\end{itemize}
The benefits and potential performance of each option needs to be further explored before a full space-based ADS-B solution can be designed.
