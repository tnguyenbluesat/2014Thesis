\section*{Declaration}
\addcontentsline{toc}{section}{Declaration}
I hereby declare that this submission is my own work and to the best of my 
knowledge it contains no materials previously published or written by another 
person, or substantial proportions of material which have been accepted for the 
award of any other degree or diploma at UNSW or any other educational 
institution, except where due acknowledgement is made in the thesis. Any 
contribution made to the research by others, with whom I have worked at 
UNSW or elsewhere, is explicitly acknowledged in the thesis. I also declare that 
the intellectual content of this thesis is the product of my own work, except to 
the extent that assistance from others in the project's design and conception or 
in style, presentation and linguistic expression is acknowledged. \vspace{1cm}
\hrule \vspace{1cm}
Thien Nguyen\\
3288816 \\
\today

\newpage
\section*{Abstract}
\addcontentsline{toc}{section}{Abstract}
\begin{comment}
Automatic Dependant Surveillance-Broadcast (ADS-B) is quickly becoming the primary method that Air Navigation Service Providers (ANSPs) and Air Traffic Control (ATC) systems use to track aircraft during flight. ADS-B requires space based receiving stations in order to track aircraft over regions where ground-based stations cannot be installed, for example, over oceans and poles. Two low risk but high-cost solutions have been proposed as secondary payloads on the Globalstar and Iridium NEXT constellations of commercial telecommunication satellites. Hosting the service in a constellation of low-cost CubeSats will provide a more economical solution, with lower production and launch costs. The key challenge in the design of the system is balancing coverage area, revisit times and link-budgets against cost and CubeSat technological limitations. Thesis B work will take place between November 2013 and June 2014 with a focus on evaluating the performance of simulations of particular orbit configurations.
\end{comment}

Automatic Dependent Surveillance-Broadcast (ADS-B) is currently being adopted by aviation authorities around the world as the standard method for tracking aircraft during flight. ADS-B coverage is available on most of the landmass in Europe, North America, Australia and South East Asia. However, gaps in coverage exist over regions where installing ADS-B receiver stations is not economically viable or feasible, such as over oceans and poles. To close these gaps, ADS-B signals can be received and retransmitted from satellites in Low Earth Orbit (LEO). There is an increased commercial interest in  implementing ADS-B re-transmitting satellite constellations. The Iridium NEXT and second Globalstar constellations of LEO satellites that are currently  under development will provide a space based ADS-B service. Using a constellation of CubeSats provides a more economical solution, with lower production, launch and satellite replacement costs. The key challenge in the design of such system entails balancing coverage area and revisit times against cost and CubeSat technological limitations.


In this thesis we provide analysis of these trade-offs and provide an insight into requirements of such a system. We have modelled popular flights over the North Pole and Pacific and Atlantic Oceans (where ADS-B coverage is not available) in Systems Tool Kit (STK) with standard commercial ADS-B transmitters. These flight paths were analysed to determine the coverage requirements of a space based ADS-B system. Aviation safety requirements from various global authorities were researched to determine the system update rates necessary to provide a safety critical service. These requirements lay the groundwork for the systems development necessary to launch and operate an ADS-B constellation. 
