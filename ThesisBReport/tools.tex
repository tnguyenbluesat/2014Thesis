\section{Analysis Tools}
The work carried out used a combination of STK, Matlab and Microsoft Excel in order to simulate, process and display the relevant data. Observations and analysis was carried out based upon the outputs from each of the three steps in the experimental process.
\subsection{STK}
STK version 10.0.2 was used in order to simulate flights, satellite orbits and communication links over the relevant analysis period. The flight paths of interest were chosen according to the process outlined in Section \ref{sec:flight_selection} and the characteristics defined in Section \ref{tab:flightSTKParams} were programmed into STK in order to create the plane and flight models. Each satellite constellation was modelled as a collection of standard satellites, with orbital parameters as specified in Section \ref{sec:sat_testcases}. The STK Communications Toolkit was then used to simulate the ADS-B links between the modelled flights and satellites, outputting data about access times and link budget. The simulation was run continuously for the duration of the analysis period discussed in \ref{sec:analysis_period}. The raw data generated by this model is described in Table \ref{tab:STKData}.

\begin{table}[htbp]
  \centering
  \caption{Raw data generated by STK}
    \begin{tabular}{lp{10cm}}
    \toprule
    Data & Description \\
    \midrule
    Access Times & The start and stop times of periods where a particular flight can access a particular satellite. This is output as \Verb|.csv| files per test of all accesses for the flight and the satellites in the orbit of interest during the test.  \\
    Link Budget & Characteristics of the communication link established during each flight-to-satellite access. Of particular interest is the received isotropic power at the satellite. This is output as \Verb|.csv| files per test of all accesses for the flight and the satellites in the orbit of interest during the test.  \\
    \bottomrule
    \end{tabular}%
  \label{tab:STKData}%
\end{table}%

\subsection{Matlab}
Matlab was used in order to concatenate and sort the data generated by the STK simulation and output the performance data specified in Section \ref{sec:perfMetrics}. The access times \Verb|.csv| files were sorted and concatenated in order to extract the `gap' times where a flight has no access to any satellite. Periods of `access' and `gap' were averaged and the minima and maxima were extracted for further analysis and display. Similarly the link budget \Verb|.csv| files were analysed in Matlab to extract the minimum received isotropic power for each flight.

\subsection{Excel}
Finally, the data outputted from Matlab was imported into Microsoft Excel in order to easily display the data for the analysis of trends and the calculation of the system trade-off decision matrix.